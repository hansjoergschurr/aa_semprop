\documentclass{scrartcl}

\usepackage[utf8]{inputenc}
\usepackage{todonotes}

\usepackage[backend=biber]{biblatex}

% Math imports
\usepackage{amsmath}
\usepackage{amsthm}
\usepackage{amssymb}
\usepackage{bussproofs}
\usepackage{mathtools}

\addbibresource{ref.bib}
\KOMAoption{bibliography}{totoc}

\title{AA}
\author{Hans-Jörg Schurr}

\begin{document}
\maketitle
\tableofcontents

\section{Problem Description}
\label{problemdesc}
This report summarizes the result of project work done for the lecture ``Abstract Argumentation'' at TU~Wien by
Stefan Woltran. The goal of the project was to empiricaly evaluate certain properties of the extensions of a set
of argumentation frameworks.

Section~\ref{problemdesc} outlines the concept of argumentation frameworks, and
extensions. Subsequently the terms Rejected Arguments and Implicit Conflicts
are defined.

Section~\ref{tools} presents the two tool that were developed for this project.
A detailed instruction on how to compile the programs, and their command line
arguments is given.

The following section describes the empirical evaluation. First the set of
frameworks is described.  Some frameworks were generated with the help of a
benchmark generator, some of them were selected from the ICCMA'15 benchmark
set. Secondly, the Aspartix framework, which was used to find the extensions
is described. Then the results are presented.

Finnaly section~\ref{conclusion} concludes this report with some final remarks.

\subsection{Abstract Argumentation}

Abstract Argumentation were introduced by Dung in 1995 \cite{dung1995}. The definitions given
here follow \cite{linsbichler2015hidden}.

First and foremost a contably infinite domain $\mathfrak{A}$. An
argumenentation framework is a tuple $F = (A,R)$ where $A$ is a finte subset of
$\mathfrak{A}$ and $R \subseteq A\times A$.  $A$ is called the set of arguments
and $R$ the attack relation.
For a framework $F = (B, S)$ the notation $A_F$
and $R_F$ is used to refer to $B$ and $S$. Futhermore $S \rightarrowtail_F a$
denotes that there is an $s \in S$ such hat $(s, a) \in R_F$. $a
\rightarrowtail_F S$ is defined equally. Intuitively the elements of $A$ are arguments of some form.
One example would be natural language sentences such as $a=$``I will eat ice cream, because the
sun is shining.'', and $b=$``The foreacast predicts it will rain in 15 minutes.''. The scond sentence
would then attack the first one and therefore $b \rightarrowtail_F a$.

 An argument $a$ is said to be
\emph{defended} by a set $S$, if for every $b \in A_F$ where $b \rightarrowtail_F a$
there is a $c \in S$ such that $c \rightarrowtail_F b$. A set $T$ is said to be \emph{defended}
by $S$ if every $a \in T$ is defended by $S$. The \emph{range} $S_F^{+}$ of a
set $S$ is the set $S \cup\{b | S \rightarrowtail_F b\}$

A \emph{semantic} is a mapping from a framework to a set $\sigma(A)$ of subsets of $A$.
$S$ then is usually called a ($\sigma$-)extension. The semamntics used in this
work are:
\begin{enumerate}
  \item Conflict-free: $S \in cf(F)$, if there is no pair $a,b \in S$, such that $(a,b) \in R$.
  \item Admissible: $S \in adm(F)$, if $S$ defends itself.
  \item Naive: $S \in nai(F)$ if $\nexists T\in cf(F)$ with $T \supset S$.
  \item Stable: $S \in stb(F)$, if $S_F^{+} = A_F$.
  \item Prefered: $S \in prf(F)$, if $S \in adm(F)$ and $\nexists T\in adm(F)$ with $T \supset S$.
  \item Stage: $S \in stg(F)$, if $\nexists T\in cf(F)$ with $T_F^{+} \supset S_F^{+}$.
  \item Semi-stable: $S \in sem(F)$, if $S \in adm(F)$ and $\nexists T\in adm(F)$ with
          $T_F^{+} \supset S_F^{+}$.
\end{enumerate}

\subsection{Framework properties}

This project concentrated on two propiertes of an argumentation framework
together with a set of extensions.


\subsection{Empirical evaluation}
\todo[inline]{write description of the evaluation}

\section{The Tools}
\label{tools}
\todo[inline]{write short description of tools and paragraphs}
\subsection{Building the tools}
\todo[inline]{write desc. on how to compile the tools}
\subsection{Analyze}
\todo[inline]{write desc. and manual of analyze}
\subsection{Statistics}
\todo[inline]{write desc. and manual of statistics}

\section{Evaluation}
\label{evaluation}
\todo[inline]{write short desc of evaluation}
\subsection{Generating the benchmarks}
\todo[inline]{Do this!}
\subsection{Finding Extensions using Aspartix}

\section{Conclusion}
\label{conclusion}
\todo[inline]{write}

\printbibliography
\end{document}
